\documentclass{article}
\oddsidemargin 0in \evensidemargin 0in \topmargin 0in
\columnsep 10pt \columnseprule 0pt 
\marginparwidth 90pt \marginparsep 11pt \marginparpush 5pt 
\headheight 0pt \headsep 0pt 
%\footheight 0pt \footskip 0pt 
\textheight 9.0in \textwidth 6.5in

\usepackage{graphicx}

\begin{document}
\hspace{2in} Name: \underline{SOLUTION}\\

\begin{center}
{\large STATISTICS 201 - Written Homework Chapter 6}\\[3mm]


\end{center}

\begin{enumerate}
\item {\em (3pts)} For the Somerset neighborhood 2009 Ames sales data tabulate the correlations between the SalePrice, LotArea, LivingArea and GarageArea. (See the textbook for an example of how to do this. Be sure to round numbers appropriately.)

%\includegraphics[width=5in]{somerset-cor.png}

\begin{tabular}{l|rrrr}
 & SalePrice & LotArea & LivingArea & GarageArea \\\hline
SalePrice & 1.00 & 0.54 & 0.63 & 0.55 \\
LotArea & & 1.00 & 0.31 & 0.62 \\
LivingArea & & & 1.00 & 0.50 \\
\end{tabular}

\item {\em (2pts)} Which two variables have the strongest correlation? What is it?

{\em SalePrice and LivingArea}

\item {\em (5pts)} Make scatterplots of each of the pairs of variables (don't hand these in). Write a paragraph briefly describing the associations, and discussing whether correlation is  or isn't an appropriate summary.

{\em Based on the scatterplots, we can say that generally there is positive association between all of these variables. If the lot area is bigger then so is the living area and garage area, and also the sale price of the home. There are some patterns between some of the variables which would make correlation not the best summary of the association. For example, between sale price and lot area we can see clusters, and within these clusters the association varies a little. There are also some outliers which may affect the correlation, for example, one house that has a sale price close to \$500000, much higher than all of the others in the data set. }

\end{enumerate}

\end{document}
