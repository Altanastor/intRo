\documentclass{article}[11pt]
\oddsidemargin 0in \evensidemargin 0in \topmargin 0in
\columnsep 10pt \columnseprule 0pt 
\marginparwidth 90pt \marginparsep 11pt \marginparpush 5pt 
\headheight 0pt \headsep 0pt 
%\footheight 0pt \footskip 0pt 
\textheight 9.0in \textwidth 6.5in

\begin{document}
\begin{center}
{\Large STATISTICS 201 - Written Homework Chapter 3 Solution}\\[3mm]
\end{center}

\noindent 
This assignment is worth a total of 35 points.

\begin{enumerate}

\item (8 pts) Michael Jordan is highly renowned basketball player who played for 15 seasons in the NBA.  The following table contains the average points per game scored by Jordan in each of his 15 seasons.

\begin{center}
\begin{tabular}{|l|c||l|c||l|c|} \hline
Year & Avg Pts/Game & Year & Avg Pts/Game & Year & Avg Pts/Game 
\\ \hline
1984 & 28 & 1989 & 34 & 1995 & 30 \\
1985 & 23 & 1990 & 32 & 1996 & 30 \\
1986 & 37 & 1991 & 30 & 1997 & 29 \\
1987 & 35 & 1992 & 33 & 2001 & 23 \\
1988 & 33 & 1994 & 27 & 2002 & 20 \\ \hline
\end{tabular}
\end{center}

\begin{enumerate}
\item (5 pts) Make a stem and leaf display of the average number of points per game that Michael Jordan scored per season.  Use a split stem. \\
\begin{center}
\begin{tabular}{c|cccccccc}
2 & 0 & 3 & 3 \\ 
2 & 7 & 8 & 9 \\
3 & 0 & 0 & 0 & 2 & 3 & 3 & 4 \\
3 & 5 & 7 \\ 
\end{tabular}
\end{center}

\item (3 pts) Using your stem and leaf display, describe the shape of this distribution.

Unimodal, skewed left and no apparent outliers.
\end{enumerate}

\item (27 pts) For this part of the assignment you will be using data collected on the population of each state in the United States in 1975.  The state populations can be found in the JMP data file {\bf StatePop1975.JMP} in the Chapter 3 Learning Module in the Unit 1 content area in Blackboard. (Note: These data are recorded by the thousands of people, so a value of 850 should be read as 850,000 people.)

\begin{enumerate}
\item (1 pt) Describe the Who for this quantitative variable.

The 50 States

\item (5 pts) Use JMP to obtain the distribution of the state populations.  Change the display to a Horizontal Layout, add a Count Axis to the histogram, and a Stem-and-Leaf display to the output.  Print out the output and turn it in with this assignment.

See the attached JMP output.

\item (2 pts) What percentage of the 50 states had between 800 thousand and 900 thousand residents in 1975?

3/50 = 6\% 

\item (2 pts) What percentage of the 50 states had greater than 10 million residents in 1975?

6/50 = 12\% 

\item (3 pts) Describe the shape of the distribution of this variable.

Unimodal, Skewed Right, two possible outliers

\item (5 pts) Give the five number summary for this variable.

The 5 number summary in this case is: Min = 365,000; Q1 = 1,026,250; Median = 2,838,500; Q3 = 5,064,000; Max = 21,198,000. 

\item (2 pts) Calculate the range and IQR for this variable.

Range = Max - Min = 21,198,000 - 365,000 = 20,833,000 \hspace{0.25in} IQR = Q3 - Q1 = 5,064,000 - 1,026,250 = 4,037,750.

\item (2 pts) Give the mean and standard deviation for this variable.

$\overline{y} = $4,246,420; $s = $4,464,490.

\item (2 pts) Explain why there is a difference of approximately 1.4 million people in the mean and median for this variable.

This is because the plot is skewed right, which causes the mean to be dragged upward, but leaves the median unaffected.

\item (3 pts) Which numerical summaries are the most appropriate for this distribution - the 5-number summary or the mean and standard deviation?  Circle the appropriate answer below.  Then explain your answer.

The correct answer is: The 5-number summary

Explain your answer: The 5-number summary is more appropriate because the distribution is heavily skewed right and has outliers.

\end{enumerate}

\end{enumerate}

\end{document}
